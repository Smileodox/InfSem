\section{Einführung}

In unserer modernen, schnelllebigen Welt ist Stress zu einer allgegenwärtigen Erscheinung geworden. Angesichts der wachsenden Anforderungen im beruflichen und privaten Leben, einem Rückgang des Wirtschaftswachstums in der westlichen Welt und angespannten geopolitischen Verhältnissen erleben viele Menschen regelmäßig hohe Stresslevel, die zunehmend mit langfristigen gesundheitlichen Problemen, insbesondere Depressionen, in Verbindung gebracht werden \cite{Wang2008} \cite{Wang2001}.

Die Bürde psychischer Störungen war bereits vor der Covid-19-Pandemie ein prägendes Thema in der Gesellschaft, doch durch die Pandemie nahm das Ausmaß dieser Probleme noch weiter zu. Die Auswirkungen der globalen COVID-19-Pandemie haben weltweit zu einer signifikanten Zunahme von Depressionen und Angststörungen geführt. Laut einer Studie wurden während der Pandemie erhebliche Einflüsse von täglichen SARS-CoV-2-Infektionsraten auf die Prävalenz von schweren depressiven Störungen und Angststörungen beobachtet. Global gesehen stieg die Anzahl der Fälle von schweren depressiven Störungen um zusätzliche 53,2 Millionen (eine Zunahme von 27,6\%), was einer Gesamtprävalenz von 3152,9 Fällen pro 100.000 Einwohner entspricht, und die Fälle von Angststörungen stiegen um 76,2 Millionen (eine Zunahme von 25,6\%), was einer Gesamtprävalenz von 4802,4 Fällen pro 100.000 Einwohner entspricht \cite{Santomauro2021}.

In dieser Zeit wachsender Herausforderungen für die psychische Gesundheit sind innovative Ansätze zur Überwachung und Früherkennung von Stresssymptomen gefragter denn je. Die grundlegenden Arbeiten von Wissenschaftlern wie Walter B. Cannon und Hans Selye haben gezeigt, dass Stress nicht nur durch subjektive Erfahrungen gekennzeichnet ist, sondern sich auch durch messbare physiologische Veränderungen auszeichnet. Cannon beschrieb die „fight-or-flight“-Reaktion, während Selye die Stressreaktion und das General Adaptation Syndrome entwickelte, die erklären, wie Stress die körperliche Gesundheit beeinflusst \cite{Cannon1915} \cite{Selye1936}.

Prognosen zufolge sollen im Jahr 2028 allein über sechshundert Millionen Wearables, wie Smartwatches und Fitness-Tracker, ausgeliefert werden \cite{IDC2023}. Diese Geräte sind bereits heute in der Lage, entscheidende physiologische Marker wie Herzfrequenz, Schlafmuster und Aktivitätslevel zu erfassen, die Schlüsselindikatoren für die Erkennung von Stress sind. Trotz der Fähigkeit, wichtige physiologische Daten zu sammeln, konzentriert sich die Forschung bisher hauptsächlich auf klinische Sensoren. Die alltäglichen Wearables, die einen kontinuierlichen Zugang zu physiologischen Daten bieten, sind jedoch in der breiten Bevölkerung bereits weit verbreitet und könnten eine Schlüsselrolle bei der Früherkennung und dem Management von Stress spielen.

In dieser Arbeit wird untersucht, wie maschinelles Lernen eingesetzt werden kann, um aus den von alltäglichen Wearables gesammelten Daten zuverlässige Indikatoren für Stress und dessen Auswirkungen auf die psychische Gesundheit zu extrahieren und zu interpretieren. Damit bietet sich die Möglichkeit, präventive Maßnahmen gegen stressbedingte Gesundheitsprobleme zu ergreifen, die direkt in das tägliche Leben integriert werden können.




\section{Theoretischer Rahmen Stress}
\subsection*{Definition Stress}
\subsection*{Zusammenhang Stress - Mental Health}
\subsection*{Physiologische Marker}

\section{Theoretischer Rahmen Wearables}
\subsection{Überblick Wearables}
\subsection{Verbaute Sensoren für obige physiologische Marker}

\section{Bisherige Ansätze Stressvorhersage ML}

\section Überführung auf Wearable Dataset

\section Diskussion und Implikationen