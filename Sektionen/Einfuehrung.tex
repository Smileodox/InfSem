\section{Einführung}

In unserer modernen, schnelllebigen Welt ist Stress zu einer allgegenwärtigen Erscheinung geworden. 
Angesichts der wachsenden Anforderungen im beruflichen und privaten Leben, einem Rückgang des Wirtschaftswachstums 
in der westlichen Welt und angespannten geopolitischen Verhältnissen erleben viele Menschen regelmäßig hohe Stresslevel, 
die zunehmend mit langfristigen gesundheitlichen Problemen, insbesondere Depressionen, in Verbindung gebracht werden \cite{Wang2008} \cite{Wang2001}.

Die Bürde psychischer Störungen war bereits vor der Covid-19-Pandemie ein prägendes Thema in der Gesellschaft, 
doch durch die Pandemie nahm das Ausmaß dieser Probleme noch weiter zu. Die Auswirkungen der globalen COVID-19-Pandemie 
haben weltweit zu einer signifikanten Zunahme von Depressionen und Angststörungen geführt. Laut einer Studie wurden 
während der Pandemie erhebliche Einflüsse von täglichen SARS-CoV-2-Infektionsraten auf die Prävalenz von schweren depressiven 
Störungen und Angststörungen beobachtet. Global gesehen stieg die Anzahl der Fälle von schweren depressiven Störungen um 
zusätzliche 53,2 Millionen (eine Zunahme von 27,6\%), was einer Gesamtprävalenz von 3152,9 Fällen pro 100.000 Einwohner entspricht, 
und die Fälle von Angststörungen stiegen um 76,2 Millionen (eine Zunahme von 25,6\%), was einer Gesamtprävalenz von 4802,4 Fällen 
pro 100.000 Einwohner entspricht \cite{Santomauro2021}.

In dieser Zeit wachsender Herausforderungen für die psychische Gesundheit sind innovative Ansätze zur Überwachung und Früherkennung 
von Stresssymptomen gefragter denn je. Die grundlegenden Arbeiten von Wissenschaftlern wie Walter B. Cannon und Hans Selye haben gezeigt, 
dass Stress nicht nur durch subjektive Erfahrungen gekennzeichnet ist, sondern sich auch durch messbare physiologische Veränderungen auszeichnet. 
Cannon beschrieb die „fight-or-flight“-Reaktion, während Selye die Stressreaktion und das General Adaptation Syndrome entwickelte, 
die erklären, wie Stress die körperliche Gesundheit beeinflusst \cite{Cannon1915} \cite{Selye1936}.

Prognosen zufolge sollen im Jahr 2028 allein über sechshundert Millionen Wearables, wie Smartwatches und Fitness-Tracker, 
ausgeliefert werden \cite{IDC2023}. Diese Geräte sind bereits heute in der Lage, entscheidende physiologische Marker wie Herzfrequenz, 
Schlafmuster und Aktivitätslevel zu erfassen, die Schlüsselindikatoren für die Erkennung von Stress sind. Trotz der Fähigkeit, 
wichtige physiologische Daten zu sammeln, konzentriert sich die Forschung bisher hauptsächlich auf klinische Sensoren. Die alltäglichen Wearables, 
die einen kontinuierlichen Zugang zu physiologischen Daten bieten, sind jedoch in der breiten Bevölkerung bereits weit verbreitet und könnten eine 
Schlüsselrolle bei der Früherkennung und dem Management von Stress spielen.

In dieser Arbeit wird untersucht, wie maschinelles Lernen eingesetzt werden kann, um physiologische Daten aus der in alltäglichen Wearables verbauten Sensorik zur Klassifizierung
von Stresszuständen zu nutzen. Dazu wird zuerst ein Überblick über die Definitionen von Stress und insbesondere dessen messbare physiologische Auswirkungen....

\section{Theoretischer Rahmen Stress}
"The most significant feature of these bodily reactions in pain and in the presence of emotion provoking objects is that they are of the nature of reflexes" \cite{Cannon1915}. 
Diese Erkenntnis bildet das Fundament von Walter B. Cannons Forschungen, die zeigen, dass der Körper in als stressvoll wahrgenommenen Situationen, wie bei Schmerz und Hunger, 
Adrenalin ausschüttet. Cannon prägte den Begriff der "Homöostase", der den Prozess beschreibt, durch den der Körper ein inneres Gleichgewicht trotz externer Störungen aufrechterhält. 
Seine Arbeiten legen den Schwerpunkt auf die physiologischen Reflexe, bekannt als die "Fight-or-Flight"-Reaktion, die eine zentrale Rolle in der Aufrechterhaltung dieser Homöostase spielen.

Hans Selye, der in den 1930er Jahren den Begriff "Stress" wissenschaftlich prägte, erweiterte dieses Konzept durch die Einführung des General Adaptation Syndrome (GAS). Er definierte Stress als eine unspezifische Antwort des 
Körpers auf jegliche Art von Reizen, ob positiv oder negativ. Das GAS beschreibt, wie der Körper in drei Phasen auf Stressoren reagiert:

Alarmreaktion: Der Körper reagiert sofort auf einen Stressor, indem er Adrenalin freisetzt und den von Cannon beschriebenen "Fight-or-Flight"-Reflex auslöst, um die 
Homöostase schnell wiederherzustellen.
Resistenz-Phase: Bei anhaltendem Stress versucht der Körper, sich anzupassen und die Homöostase durch Regulation physiologischer und psychologischer 
Systeme aufrechtzuerhalten.
Erschöpfungs-Phase: Gelingt es dem Körper nicht, sich anzupassen, führt dies zu einer Erschöpfung, die gesundheitliche Schäden bis hin zum Tod zur Folge 
haben kann \cite{Selye1936}.

\subsection{Moderne Interpretation}

Aufbauend auf diesen grundlegenden Konzepten entwickelt sich die Definition von Stress aber durchaus mit wachsendem Forschungsstand weiter. 
In neueren Konzepten, wie von David S. Goldstein und anderen Wissenschaftlern entwickelt, wird Stress als eine bewusste oder unbewusste Bedrohung der 
Homöostase angesehen. Dieser Ansatz erkennt an, dass die Reaktion des Körpers auf Stress durchaus spezifisch sein kann, je nachdem, welche Herausforderung 
der Homöostase vorliegt, wie das Individuum den Stressor wahrnimmt und wie gut es sich in der Lage sieht, mit ihm umzugehen.

Goldstein hebt hervor, dass neben Cannons "sympathoadrenerges System", welches sich Allgemein auf das Zusammenspiel des Sympatischen Nervensystems 
und der Nebenniere zur wiederherstellung der Homöostase durch hormonelle Regulierung bezieht, zur Aufrechterhaltung der Homöostase auch das \ac{HPA-Achse}
eine zentrale Rolle in der Stressreaktion spielt. Beide Systeme sind aktiv an der Regulation und Anpassung bei Stress beteiligt und tragen dazu bei, eine 
neue Stabilität zu erreichen, die als "Allostase" bezeichnet wird.

Goldstein benutzt eine Analogie zu einem Heizkreislauf, um Allostase zu erklären: Ein Mensch stellt in einem Thermostat 
eine Soll-Temperatur ein; dieses Thermostat erfasst die aktuelle Raumtemperatur und aktiviert die Heizung, wenn die Temperatur unter den Sollwert fällt. 
Ähnlich reguliert der Körper durch Allostase die "innere Temperatur" oder physiologische Sollwerte, indem er nach Bedarf Anpassungen vornimmt, 
um Stabilität durch Veränderung zu erreichen.

Allostase beschreibt die Fähigkeit, Stabilität durch Veränderung zu erhalten, wobei "allostatische Last" die Kosten dieser Anpassungen misst. 
Diese Last kann, besonders bei langfristigem oder intensivem Stress, zu negativen gesundheitlichen Auswirkungen führen. Goldstein führt aus, 
dass zum Beispiel eine chronische Erhöhung des Blutdrucks, die zunächst dazu dient, die Durchblutung des Gehirns sicherzustellen, langfristig zu 
Gefäßschädigungen und daraus resultierenden Krankheiten wie Schlaganfall oder Herzinfarkt führen kann. \cite{Gold2007}

Das moderne Verständnis von Stress betont somit nicht nur die Akutreaktion des Körpers, sondern auch die langfristigen Auswirkungen der Stressbewältigung 
und die Bedeutung der Anpassungsmechanismen, die über die unmittelbare "Fight-or-Flight"-Reaktion hinausgehen.

\subsection{Physiologische Marker für Stress}

Wie bereits erleuchtet wurde, versucht der Körper verallgemeinert über zwei Systeme mit Stress umzugehen; der \ac{HPA-Achse} und dem sympathoadrenergem System. 
Das sympathoadrenerge System bewirkt eine Auscchüttung von Adrenalin sowie Noradrenalien, welche in kurzer Zeit physiologische Anpassung wie einer Erhöhung des Blutdrucks bewirken,
wohingegen die \ac{HPA-Achse} etwas langsamer über Anregung von Cortisolproduktion fungiert. \cite{Kaiser2023}

Da sich diese Arbeit darauf fokusiert, Stress anhand messbarer physiologischer Merkmale zu klassifizieren, widmen wir einen etwas ausführlicheren Blick auf ein 




\subsection{Definition Stress}
h
\subsection{Zusammenhang Stress - Mental Health}
h
\subsection{Physiologische Marker}
h
\section{Theoretischer Rahmen Wearables}
h
\subsection{Überblick Wearables}
h
\subsection{Verbaute Sensoren für obige physiologische Marker}
h
\section{Bisherige Ansätze Stressvorhersage ML}
h
\section{Überführung auf Wearable Dataset}
h
\section{Diskussion und Implikationen}
h